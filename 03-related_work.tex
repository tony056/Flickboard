% ############################################################
% RELATED WORK
% ############################################################

\section{Related Work}
%\subsection{Co-located Keyboard and Touchpad}
Previous research has shown that co-locating two devices will improve user performance \cite{TouchNType}.
ThumbSense \cite{ThumbSense} helps users keep their fingers on the home row by using keyboard keys as mouse buttons when it detects a movement of the thumb on the touchpad.
%Longpad\cite{longpad} has shown that a larger touchpad occupies the whole area below keyboard can enable more possibilities for interactions.
Type–Hover–Swipe \cite {96bytes} implemented a modified keyboard with infra-red proximity sensors that recognizes in-air hand gestures and obtains coarse finger positions. The depth map generated by the infrared range finder is fast and stable, but the finger positions obtained by the system are too rough to control a mouse cursor because the sensors were interspersed between the key caps.
DGTS \cite{DTGS} uses capacitive sensing technology, in contrast, to obtain a higher resolution image to control the cursor. However, the integrated device still requires manual mode switching to avoid false triggering of the pointing device.
